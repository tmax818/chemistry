\documentclass[12pt]{article}
\usepackage{geometry}
\geometry{margin=2cm}
\usepackage{titlesec}
\usepackage{enumitem}
\usepackage{graphicx}
\usepackage{hyperref}
\usepackage{fancyhdr}
\usepackage{multicol}

\pagestyle{fancy}
\fancyhf{}
\rhead{Chemistry AB}
\lhead{Syllabus 2025–2026}
\rfoot{Page \thepage}

\titleformat{\section}{\large\bfseries}{\thesection}{1em}{}
\titleformat{\subsection}{\normalsize\bfseries}{\thesubsection}{1em}{}

\title{\vspace{-1cm}\textbf{Course Syllabus}\\\large 2025–2026}
\author{}
\date{}

\begin{document}

\maketitle
\vspace{-5em}

\section*{Course Information}
\begin{tabular}{ll}
\textbf{course:} & Chemistry AB \\
\textbf{prerequisite:} & Algebra 1AB or equivalent \\
\textbf{corequisite:} & Geometry 1AB is recommended\\
\end{tabular}


\section*{Instructor Information}
\begin{tabular}{ll}
\textbf{name:} & Mr. Tyler Maxwell \\
\textbf{email:} & tyler.maxwell@lausd.net \\
\textbf{room:} & 305E \\
\textbf{office hours:} & By appointment
\end{tabular}

\section*{Course Description}
Chemistry is a laboratory-based college-preparatory course. Laboratory experiments provide the empirical basis for understanding and confirming concepts. This course emphasizes discussions, activities, and laboratory exercises, which promote the understanding of the behavior of matter at the macroscopic and the molecular-atomic levels. Chemical principles are introduced so that students will be able to explain the composition and chemical behavior of their world. Chemistry AB lays the foundation for further studies in Chemistry and also serves as an Advanced Placement Chemistry readiness course. 

\textbf{Chemistry AB meets the Grades 9-12 District physical science requirement. Students must complete one physical and one life science requirement. This course meets one year of the University of California 'd' entrance requirement for laboratory science.}


\section*{Textbook and Materials}
\begin{itemize}
    \item \textit{Chemistry: Matter and Change}, Glencoe/McGraw-Hill
    \item Composition notebook for lab reports
\end{itemize}

\section*{Classroom Expectations}
\begin{itemize}
    \item Be respectful and prepared.
    \item Follow all safety rules during labs.
    \item Submit assignments on time.
    \item Collaborate honestly—no plagiarism or cheating.
\end{itemize}


\section{Course Objectives}
\subsection{The Periodic Table}

\begin{itemize}
    \item Students know how to use the periodic table to identify metals, semimetals, nonmetals, and halogens.
    \item Students know how to use the periodic table to identify the lanthanide, actinide, and transactinide elements and know that the transuranium elements were synthesized and identified in laboratory experiments through the use of nuclear accelerators.
    \item Students know how to use the periodic table to identify alkali metals, alkaline earth metals and transition metals, trends in ionization energy, electronegativity, and the relative sizes of ions and atoms.
\end{itemize}


\subsection{Atomic Structure}

\begin{itemize}
    \item Students know the experimental basis for Thomson's discovery of the electron, Rutherford's nuclear atom, Millikan's oil drop experiment, and Einstein's explanation of the photoelectric effect.
    \item Students know the experimental basis for the development of the quantum theory of atomic structure and the historical importance of the Bohr model of the atom.
    \item Students know that spectral lines are the result of transitions of electrons between energy levels and that these lines correspond to photons with a frequency related to the energy spacing between levels by using Planck's relationship ($E = hv$).
    \item Students know the nucleus of the atom is much smaller than the atom yet contains most of its mass
\end{itemize}


\subsection{Periodicity and Electron Arrangement}

\begin{itemize}
    \item Students know how to relate the position of an element in the periodic table to its atomic number and atomic mass.
    \item Students know how to relate the position of an element in the periodic table to its quantum electron configuration and to its reactivity with other elements in the table.
    \item Students know how to use the periodic table to determine the number of electrons available for bonding.
\end{itemize}

\subsection{Chemical Bonding}

\subsection{The Mole Concept}

\subsection{Stoichiometry}

\subsection{Kinetic Motion of Gases}

\subsection{The Gas Laws}

\subsection{Solutions}

\subsection{Chemical Equilibrium}

\subsection{Acids and Bases}

\subsection{Acid/Base Equilibrium}

\subsection{Chemical Thermodynamics}

\subsection{Organic Chemistry}

\subsection{Nuclear Chemistry}

\subsection{Nuclear Energy}

\subsection{Particle Physics}


\section*{Grading Policy}
\begin{tabular}{ll}
Homework & 15\% \\
Labs and Lab Reports & 25\% \\
Quizzes & 20\% \\
Tests and Projects & 30\% \\
Participation & 10\% \\
\end{tabular}

\section*{Late Work and Make-up Policy}
Late assignments will receive a 10\% deduction per day late, up to 50\%. Make-up work is allowed for excused absences and must be completed within one week of return.

\section*{Contact and Communication}
Students and parents are encouraged to email with any questions or concerns. Updates, assignments, and additional resources will be posted on the Google Classroom portal.

\end{document}
