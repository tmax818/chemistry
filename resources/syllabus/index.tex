% Created 2025-05-08 Thu 20:29
% Intended LaTeX compiler: pdflatex
\documentclass[11pt]{article}
\usepackage[utf8]{inputenc}
\usepackage[T1]{fontenc}
\usepackage{graphicx}
\usepackage{longtable}
\usepackage{wrapfig}
\usepackage{rotating}
\usepackage[normalem]{ulem}
\usepackage{amsmath}
\usepackage{amssymb}
\usepackage{capt-of}
\usepackage{hyperref}
\author{Mr. Maxwell}
\date{\today}
\title{Chemistry Syllabus}
\hypersetup{
 pdfauthor={Mr. Maxwell},
 pdftitle={Chemistry Syllabus},
 pdfkeywords={},
 pdfsubject={},
 pdfcreator={Emacs 30.1 (Org mode 9.7.11)}, 
 pdflang={English}}
\begin{document}

\maketitle
\tableofcontents

\section{Chemistry}
\label{sec:orgf76686c}

Instructor: Tyler Maxwell
Email: \href{mailto:tyler.maxwell@lausd.net}{tyler.maxwell@lausd.net}  
Classroom: E305  
Office Hours: After school, by appointment  


This course explores the principles of chemistry through hands-on activities, labs, and discussions. The syllabus is aligned with \href{https://www.cde.ca.gov/ci/pl/documents/ngsshsphyicalscidci.pdf}{California State Standards} and \href{https://nap.nationalacademies.org/catalog/18290/next-generation-science-standards-for-states-by-states}{NGSS}.
\section{Course Content}
\label{sec:orgc7fcb9b}

\subsection{The periodic table displays the elements in increasing atomic number and shows how periodicity of the physical and chemical properties of the elements relates to atomic structure.}
\label{sec:orgd1c971c}

\begin{itemize}
\item use the periodic table to identify metals, semimetals, nonmetals, and halogens.
\item use the periodic table to identify alkali metals, alkaline earth metals and transition metals, trends in ionization \textbf{energy}, electronegativity, and the relative sizes of ions and atoms.
\item use the periodic table to determine the number of electrons available for bonding.
\item use the periodic table to identify the lanthanide, actinide, and transactinide elements and know that the transuranium elements were synthesized and identified in laboratory experiments through the use of nuclear accelerators.
\item relate the position of an element in the periodic table to its \textbf{\textbf{atomic number}} and \textbf{atomic mass}.
\item relate the position of an element in the periodic table to its quantum electron configuration and to its reactivity with other elements in the table.

\item the experimental basis for *Thomson*’s discovery of the electron, *Rutherford*’s nuclear atom, *Millikan*’s oil drop experiment, and Einstein’s explanation of the photoelectric effect.
\item the experimental basis for the development of the quantum theory of atomic structure and the historical importance of the \textbf{Bohr model} of the atom.
\end{itemize}


\begin{itemize}
\item \textbf{spectral lines} are the result of transitions of electrons between \textbf{*energy} levels* and that these lines correspond to \textbf{photons} with a \textbf{frequency} related to the \textbf{energy} spacing between levels by using *Planck*’s relationship (\(E = hv\)).
\item the \textbf{nucleus} of the atom is much smaller than the atom yet contains most of its mass.
\end{itemize}
\subsection{Biological, chemical, and physical properties of matter result from the ability of atoms to form bonds from electrostatic forces between electrons and *proton*s and between atoms and *molecule*s.}
\label{sec:org844fd44}

\begin{itemize}
\item atoms combine to form \textbf{molecule*s by sharing electrons to form *covalent} or \textbf{metallic} bonds or by exchanging electrons to form \textbf{ionic} bonds.
\item \textbf{chemical bonds} between atoms in *molecule*s such as \(H_2\), \(CH_4\), \(NH_3\), \(H_{2}CCH_2\), \(N_2\), \(Cl_2\), and many large biological *molecule*s are covalent.
\item salt crystals, such as NaCl, are repeating patterns of positive and negative ions held together by electrostatic attraction.
\item the atoms and *molecule*s in liquids move in a random pattern relative to one another because the intermolecular forces are too weak to hold the atoms or *molecule*s in a solid form.
\item draw \textbf{Lewis dot structures}.
\item predict the shape of simple \textbf{molecule*s and their polarity from *Lewis dot structures}.
\item how \textbf{electronegativity} and \textbf{ionization \textbf{energy}} relate to bond formation.
\item identify solids and liquids held together by \textbf{Van der Waals} forces or \textbf{hydrogen bonding} and relate these forces to volatility and boiling/melting point *temperature*s.
\end{itemize}
\subsection{The conservation of atoms in chemical reactions leads to the principle of conservation of matter and the ability to calculate the mass of *product*s and *reactant*s.}
\label{sec:org5b495e3}

\begin{itemize}
\item describe chemical reactions by writing balanced equations.
\item the quantity one \textbf{mole} is set by defining one mole of carbon 12 atoms to have a mass of exactly 12 grams.
\item one mole equals \(6.02 \times 10^{23}\) particles (atoms or *molecule*s).
\item determine the \textbf{molar mass} of a \textbf{molecule} from its chemical formula and a table of atomic masses and how to convert the mass of a molecular substance to moles, number of particles, or \textbf{volume} of gas at standard \textbf{temperature} and \textbf{pressure}.
\item calculate the masses of *reactant*s and *product*s in a chemical reaction from the mass of one of the *reactant*s or *product*s and the relevant atomic masses.
\item calculate \textbf{percent yield} in a chemical reaction.
\item identify reactions that involve \textbf{oxidation and reduction} and how to balance oxidation-reduction reactions.
\end{itemize}
\subsection{The \textbf{kinetic} molecular theory describes the motion of atoms and *molecule*s and explains the properties of gases.}
\label{sec:orgc1b0d46}

\begin{itemize}
\item the random motion of \textbf{molecule*s and their collisions with a surface create the observable *pressure} on that surface.
\item the random motion of *molecule*s explains the diffusion of gases.

\item apply the \textbf{gas laws} to relations between the \textbf{pressure}, \textbf{temperature}, and \textbf{volume} of any amount of an \textbf{ideal gas} or any mixture of *ideal gas*es.
\item the values and meanings of standard \textbf{temperature} and \textbf{pressure} (STP).
\item convert between the \textbf{Celsius} and \textbf{Kelvin} \textbf{temperature} scales.
\item there is no \textbf{temperature} lower than 0 \textbf{Kelvin}.
\item the \textbf{kinetic} theory of gases relates the absolute temperature of a gas to the \textbf{average *kinetic} \textbf{energy*} of its *molecule*s or atoms.
\item solve problems by using the \textbf{ideal gas} law in the form \(PV = nRT\).
\item apply \textbf{Dalton’s law of partial *pressure*s} to describe the composition of gases and \textbf{Graham’s law} to predict \textbf{diffusion} of gases.
\end{itemize}
\subsection{*Acid*s, *base*s, and salts are three classes of compounds that form ions in water solutions.}
\label{sec:orgf474de6}

\begin{itemize}
\item the observable properties of *acid*s, *base*s, and salt solutions.
\item *acid*s are hydrogen-ion-donating and *base*s are hydrogen-ion-accepting substances.
\item strong *acid*s and *base*s fully dissociate and weak *acid*s and *base*s partially dissociate.
\item use the \textbf{pH} scale to characterize \textbf{acid} and \textbf{base} solutions.
\item the \textbf{Arrhenius}, \textbf{Brønsted-Lowry}, and \textbf{Lewis} \textbf{acid*–*base} definitions.
\item calculate \textbf{pH} from the hydrogen-ion \textbf{concentration}.
\item \textbf{buffers} stabilize \textbf{pH} in \textbf{acid*–*base} reactions.
\end{itemize}
\subsection{Solutions are homogenous mixtures of two or more substances.}
\label{sec:org56714c7}

\begin{itemize}
\item the definitions of \textbf{solute} and \textbf{solvent}.
\item describe the dissolving process at the molecular level by using the concept of \textbf{random molecular motion.}
\item temperature, \textbf{pressure}, and surface area affect the dissolving process.
\item calculate the \textbf{\textbf{concentration}} of a solute in terms of \textbf{grams per liter}, \textbf{molarity}, \textbf{parts per million}, and \textbf{percent composition}.
\item the relationship between the \textbf{molality} of a solute in a solution and the solution’s \textbf{depressed freezing point} or \textbf{elevated boiling point}.
\item how \textbf{molecule*s in a solution are separated or purified by the methods of *chromatography} and \textbf{distillation}.
\end{itemize}
\subsection{\textbf{Energy} is exchanged or transformed in all chemical reactions and physical changes of matter.}
\label{sec:org7903a28}

\begin{itemize}
\item describe \textbf{temperature} and \textbf{heat flow} in terms of the motion of *molecule*s (oratoms).
\item chemical processes can either release (\textbf{exothermic}) or absorb (\textbf{endothermic}) thermal \textbf{energy}.
\item \textbf{energy} is released when a material condenses or freezes and is absorbed when a material evaporates or melts.
\item solve problems involving \textbf{heat flow} and \textbf{temperature} changes, using known values of \textbf{specific heat} and \textbf{latent heat} of phase change.
\item apply \textbf{Hess’s law} to calculate \textbf{enthalpy} change in a reaction.
\item use the \textbf{Gibbs free energy equation} to determine whether a reaction would be \textbf{spontaneous}.
\end{itemize}
\subsection{Chemical reaction rates depend on factors that influence the frequency of collision of \textbf{reactant} *molecule*s.}
\label{sec:org80cfafa}

\begin{itemize}
\item the rate of reaction is the decrease in \textbf{concentration} of \textbf{reactant*s or the increase in *concentration} of *product*s with time.
\item how reaction rates depend on such factors as \textbf{concentration}, \textbf{temperature}, and \textbf{pressure}.
\item the role a \textbf{catalyst} plays in increasing the reaction rate.
\item the definition and role of activation \textbf{energy} in a chemical reaction.
\end{itemize}
\subsection{Chemical \textbf{equilibrium} is a dynamic process at the molecular level.}
\label{sec:orgf64e8cc}

\begin{itemize}
\item use \textbf{LeChatelier’s principle} to predict the effect of changes in \textbf{concentration}, \textbf{temperature}, and \textbf{pressure}.
\item \textbf{equilibrium} is established when forward and reverse reaction rates are equal.
\item write and calculate an \textbf{equilibrium} constant expression for a reaction.
\end{itemize}
\subsection{The bonding characteristics of carbon allow the formation of many different organic *molecule*s of varied sizes, shapes, and chemical properties and provide the biochemical basis of life.}
\label{sec:orgcde216e}

\begin{itemize}
\item large \textbf{molecules} (polymers), such as \textbf{proteins}, \textbf{nucleic acids}, and \textbf{starch}, are formed by repetitive combinations of simple subunits.
\item the bonding characteristics of carbon that result in the formation of a large variety of structures ranging from simple hydrocarbons to complex polymers and biological \textbf{molecules}.
\item amino *acid*s are the building blocks of proteins.
\item the system for naming the ten simplest linear hydrocarbons and isomers that contain single bonds, simple hydrocarbons with double and triple bonds, and simple \textbf{molecules} that contain a \textbf{benzene ring}.
\item identify the \textbf{functional groups} that form the basis of \textbf{alcohols}, \textbf{ketones}, \textbf{ethers}, \textbf{amines}, \textbf{esters}, \textbf{aldehydes}, and \textbf{organic acids}.
\item the R-group structure of \textbf{amino acids} and know how they combine to form the \textbf{polypeptide} backbone structure of proteins.
\end{itemize}
\subsection{Nuclear processes are those in which an atomic \textbf{nucleus} changes, including radioactive decay of naturally occurring and human-made \textbf{isotopes}, nuclear fission, and nuclear fusion.}
\label{sec:org76c8016}

\begin{itemize}
\item \textbf{proton*s and *neutron*s in the *nucleus} are held together by nuclear forces that overcome the \textbf{electromagnetic} repulsion between the *proton*s.
\item the \textbf{energy} release per gram of material is much larger in nuclear fusion or fission reactions than in chemical reactions. The change in mass (calculated by \(E = mc^2\) ) is small but significant in nuclear reactions.
\item some naturally occurring \textbf{isotopes} of elements are radioactive, as are \textbf{isotopes} formed in nuclear reactions.
\item the three most common forms of radioactive decay (\textbf{alpha}, \textbf{beta}, and \textbf{gamma}) and know how the \textbf{nucleus} changes in each type of decay.
\item \textbf{alpha}, \textbf{beta}, and \textbf{gamma} radiation produce different amounts and kinds of damage in matter and have different penetrations.
\item calculate the amount of a radioactive substance remaining after an integral number of half lives have passed.
\item \textbf{proton*s and *neutron*s have substructures and consist of particles called *quarks}..
\end{itemize}
\section{Assessments and Grading}
\label{sec:org1f4246f}

\begin{itemize}
\item Labs and Reports: 30\%
\item Quizzes and Exams: 40\%
\item Homework and Classwork: 20\%
\item Participation: 10\%
\end{itemize}
\section{Important Dates}
\label{sec:org031019f}

\begin{itemize}
\item \textbf{Midterm Exam:} TBD
\item \textbf{Final Exam:} TBD
\end{itemize}
\section{Materials Needed}
\label{sec:org0ecd7b5}

\begin{itemize}
\item Textbook: Chemistry in the Universe
\item Lab notebook (TODO provided).
\item Scientific calculator.
\item Safety goggles (provided in class).
\end{itemize}
\section{Class Policies}
\label{sec:orgdd2dcdb}

\begin{enumerate}
\item \textbf{Attendance:} Regular attendance is required for success.
\item \textbf{Safety:} Lab safety rules must be followed at all times.
\item \textbf{Late Work:} Assignments are due on the posted date; late submissions incur penalties unless prior arrangements are made.
\item \textbf{Academic Integrity:} Plagiarism or cheating will result in disciplinary action.
\end{enumerate}
\end{document}
