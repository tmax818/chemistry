\documentclass[12pt]{exam}
\usepackage[version=4]{mhchem}

\pagestyle{headandfoot}
\runningheadrule
\firstpageheader{Name: \fillin[name][3cm]}{Atomic Structure Notes}{Period \fillin[answer][1cm]}
\runningheader{Chemistry B}
{First Exam, Page \thepage\ of \numpages}
{\today}
\firstpagefooter{}{}{}
\runningfooter{}{}{}

\begin{document}
\twocolumn
    
\begin{questions}
        
\section{Atomic Structure}

\subsection{atomic number and mass}

\question The \fillin[atomic number][3cm] is the number of \fillin[protons][3cm] in the nucleus of an atom.

\question The \fillin[mass number][3cm] is the total number of \fillin[protons][2cm] and \fillin[neutrons][2cm]in the nucleus of an atom.



\question $$\ce{^{1}_{}H}$$
What does the 1 mean?

\fillwithlines{2cm}


\question $$\ce{^{4}_{2}He}$$

\question What does the 4 mean? 

\fillwithlines{1cm}

\question What does the 2 mean?
\fillwithlines{1cm}

\question $$\ce{^{7}_{3}Li}$$

How many protons does Lithium have? \fillin[3][2cm]

How many neutrons does Lithium have? \fillin[4][2cm]

\question $$\ce{^{2}_{}H}$$ based on this symbol, how many protons does Hydrogen have? \fillin[1][1cm]

How many neutrons? \fillin[1][1cm]

\vspace{3cm}

\subsection{The Bohr model}

\question   The Bohr Model - Bohr proposed that an atom was a nucleus with electrons "orbiting" in different \fillin[energy levels].

\question The electrons closest to the nucleus have the \fillin[lowest] energy, while those further from away have \fillin[higher] energy.

\end{questions}
\end{document}