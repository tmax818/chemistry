\documentclass[12pt,answers]{exam}
\usepackage[version=4]{mhchem}
\usepackage[usenames,dvipsnames]{color}
\usepackage[T1]{fontenc}
\usepackage{tikz}
\usetikzlibrary{positioning}
\usetikzlibrary{arrows}
\usetikzlibrary{shapes.multipart}
\usepackage[caption=false]{subfig}
\usepackage{tabularx,tikz}
\usepackage{graphicx}
\usepackage{color}
\usepackage{pdfpages}

\pagestyle{headandfoot}
\runningheadrule
\firstpageheader{Name: \fillin[][4cm]}{Atomic Structure Notes}{Period \fillin[][1cm]}
\runningheader{Chemistry B}{Atomic Structure}{\today}
\firstpagefooter{}{}{}
\runningfooter{Chemistry B}{Atomic Structure, Page \thepage\ of \numpages}{\today}

\begin{document}

    
\begin{questions}

\question What color element symbol(like He, or Li) designates elements that are a gas at a given temperature? \fillin[red][1cm]

\question  What family of elements does the red color indicate? \fillin[Transition metals][5cm]

\question Hover the cursor over Magnesium, Mg, what type of specific information is given about the element?\\ 
\fillin[Atomic number, symbol, name, atomic mass, and electron count for each shell. ][17cm]

\question  Now click on the element name for Magnesium, Mg. What information are you presented with?

\fillin[The Wikipedia page ][10cm]

\question What are the element symbols for the Metalloid elements?\\

\fillin[B, Si, Ge, As, Sb, Te, At ][6cm]

\question What is the symbol and atomic number of the element with the most protons according to this periodic table? Click on its symbol to find out what year it was discovered.\\

\fillin[Og Atomic number 118. It was discovered in 2002][12cm]

\question If an element's symbol is written in blue, what information does this tell you?\\

\fillin[That element is liquid phase based on the temperature selected by the slider ][12cm]

\question Reduce the temperature value to 0K, using the control. How many gaseous elements exist at 0K according to the table? How many liquid elements exist? List the symbol of any elements that fall in these categories.

\fillin[no gasses, He is the only liquid][6cm]

\question Change the temperature value to room temperature, 298K. At the top of the
screen, choose the “Properties” tab. How does this view differentiate between solid,
liquid and gas element states?

\question Using your cursor, choose Iron, Fe. Determine the following values for Iron:

 Melting Point = \fillin[$8111K$][3cm]\\
 Boiling Point = \fillin[$3134K$][3cm]\\
 Atomic Radius = \fillin[$156pm$][3cm]\\
 Density = \fillin[$7874 kg/m^3$][3cm]\\

\end{questions}

\end{document}