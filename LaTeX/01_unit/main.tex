\documentclass[12pt]{exam}

\usepackage{graphicx}
\usepackage{mhchem}

\pagestyle{headandfoot}
\runningheadrule
\firstpageheader{name:\fillin[][4cm]}{period:\fillin[][1cm]}{lesson 1.3: forces on matter}
\runningheader{lesson 1.3: forces on matter}
{Class Notes}
{Page \thepage\ of \numpages}
\firstpagefooter{}{}{}
\runningfooter{PACS}{Mr. Maxwell}{page \thepage\ of \numpages}


\begin{document}

\begin{questions}

    \section{Warm Up}

    \question What are the three particle that make up an atom? Which one is positive, negative, and neutral?

   \begin{center}
    \begin{tabular}{|c|c|}
        \hline
        particle & charge \\ \hline
        \hspace{2cm} & \hspace{2cm} \\ \hline
        \ & \ \\ \hline
        \ & \ \\ \hline

    \end{tabular}
   \end{center}


    \question Draw a picture of a \ce{^4_2He} atom. Label the nucleus, protons, neutrons and electrons.

    \vspace{3cm}

    \section{Forces on Matter}

    \question A force is a \fillin or a \fillin on an object.

  \begin{center}
    \includegraphics[width=0.5\textwidth]{pushpull}
  \end{center}


    \question We draw a force with an arrow that shows the \fillin of the force.
    
    
    \question There are two kinds of forces:
    \begin{parts}
        \part \fillin[][4cm] force
        \part \fillin[][4cm] force
    \end{parts}
    

\subsection{Gravitational Force}
\question Gravity is a force on the \fillin of an object caused by the mass of \fillin object.
\question Gravity is always a \fillin force between two masses.
\question The gravitational force between two masses happens no matter how \fillin the masses are from each other.
\question The gravitational force gets \fillin when the masses get farther apart.
\question On Earth the gravitational force on objects is always \fillin.

\subsection{Electromagnetic Force}

\question The electromagnetic force is caused by the pushing and pulling between the electric charges of \fillin and \fillin in an object no matter how far apart they are.
\question The electromagnetic force gets \fillin when the electric charges are farther apart.
\question Two positive charges (protons) will \fillin (repel) each other away.
\question Two negative charges (electrons) will \fillin (repel) each other away.
\question A positive charge (proton) and a negative charge (electron) will pull (\fillin) each other.

\subsection{Examples of Electromagnetic Force}

\question \fillin electric charges (protons) in a metal can pull on (\fillin) negative electric charges (electrons) in a balloon.
\question The positive (protons) and negative (\fillin) electric charges in a magnet can either push the magnets apart (\fillin) or pull them together (attract).
\question The negative electric charges (electrons) in your hand \fillin on the negative electric charges (electrons) in an object that you touch.
\question When you stretch a rubber band the protons attract the electrons and \fillin back.


\end{questions}


\pagebreak

\end{document}