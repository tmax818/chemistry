\documentclass{exam}

\usepackage{chemformula}


\pagestyle{headandfoot}
\runningheadrule
\firstpageheader{name:\fillin}{period: \fillin}{id: \fillin}
\runningheader{Chemistry A}
{First Exam, Page \thepage\ of \numpages}
{\today}
\firstpagefooter{}{}{rev. \today}
\runningfooter{}{}{}

\begin{document}


\begin{center}
    \fbox{\parbox{5.5in}{\centering
    Answer the questions to the best of your \textbf{current} knowledge! \\ Point values are \textbf{extra credit}!!}}
\end{center}


\begin{questions}
    
\question[2] Who is considered the father of modern atomic theory?
\begin{checkboxes}
    \choice J.J. Thomson 
    \choice Ernest Rutherford
    \choice John Dalton 
    \choice Niels Bohr 
\end{checkboxes}


\question[2] Which of the following statements is part of Dalton's atomic theory?
\begin{checkboxes}
    \choice Atoms can be created or destroyed.  
    \choice All atoms of an element are identical.  
    \choice Atoms are divisible into smaller particles.  
    \choice Atoms can exist in different states (solid, liquid, gas). 
\end{checkboxes}

\question[2] What subatomic particle was discovered using the cathode ray tube experiment?
\begin{checkboxes}
    \choice Proton  
    \choice Neutron  
    \choice Electron  
    \choice Photon 
\end{checkboxes}

\question[2] Rutherford's gold foil experiment led to the discovery of which atomic structure?  
\begin{checkboxes}
    \choice Electron cloud  
    \choice Nucleus  
    \choice Proton  
    \choice Neutron  
\end{checkboxes}

\question[2] Which particle has a negative charge?
\begin{checkboxes}
    \choice Proton  
    \choice Neutron  
    \choice Electron  
    \choice All of the above  
\end{checkboxes}  

\question[5] Ballance the following chemical equation:

\begin{equation}
    \ch{ H2 \;\;\;  +  \quad  O2 \quad ->  \quad H2O}
\end{equation}

\question[5] Describe the main idea of the plum pudding model proposed by J.J. Thomson.  

\makeemptybox{1in}

\question[5] Explain how the Bohr model of the atom differs from Rutherford's model.  

\makeemptybox{1in}

\question[5] What is an isotope? Provide an example.  

\makeemptybox{1in}

\question[2] The atomic mass of an element is the average mass of all its isotopes.  
\begin{checkboxes}
    \choice True
    \choice False
\end{checkboxes}


\question[2] Neutrons are positively charged particles found in the nucleus of an atom.  
\begin{checkboxes}
    \choice True
    \choice False
\end{checkboxes}

\vspace{1cm}

\question[5] How did the development of the quantum mechanical model improve upon earlier models of the atom?  





\end{questions}





\end{document}