\documentclass[12pt]{exam}
\usepackage[version=4]{mhchem}
\usepackage[usenames,dvipsnames]{color}
\usepackage[T1]{fontenc}
\usepackage{tikz}
\usetikzlibrary{positioning}
\usetikzlibrary{arrows}
\usetikzlibrary{shapes.multipart}
\usepackage[caption=false]{subfig}
\usepackage{tabularx,tikz}
\usepackage{graphicx}
\usepackage{color}
\usepackage{pdfpages}

\pagestyle{headandfoot}
\runningheadrule
\firstpageheader{Names: \fillin[][6cm]}{}{Period \fillin[][1cm]}
\runningheader{Chemistry B}{Atomic Structure}{\today}
\firstpagefooter{}{}{}
\runningfooter{Chemistry B}{Atomic Structure, Page \thepage\ of \numpages}{\today}

\begin{document}

\noindent\rule{\textwidth}{1pt}

Arrange the atoms on the table provided from largest to smallest going down the column. The column numbers should match the number of valence electrons for the atom.

\begin{questions}
  \question What do you notice about your arrangement and the periodic table?

  \vspace{4cm}

\end{questions}

\noindent\rule{\textwidth}{1pt}
\begin{questions}
 \question Balance the following equation: 
 \vspace{1cm}
 \ce{CH4} + \fillin[2][.5cm]\ce{O2} \ce{->}  \ce{CO2} + \fillin[2][.5cm]\ce{H2O}

 \question create the equation using the atoms provided. Take a picture of the reatants:\\
 \ce{CO2} + \fillin[2][.5cm]\ce{H2O}\\ and attach the photo to the schoology assignment.

 \vspace{1cm}

 Take a picture of the products:

 \ce{CH4} + \fillin[2][.5cm]\ce{O2} \\ and attach the photo to the schoology assignment.
\end{questions}

\noindent\rule{\textwidth}{1pt}

Use the atoms to create a \ce{NaCl} molecule and a \ce{H2O} molecule

\begin{questions}
  \question Disolve the salt in a beaker of water. Record your observations:

  \vspace{2cm}

  \question Observe the strength of the bonds in the model of \ce{NaCl} and the model of \ce{H2O}. Highlight or circle the molecule with the stronger bonds:

  \ce{NaCl} + \ce{H2O}
\end{questions}

\end{document}