\documentclass[a4paper,11pt]{exam}

\usepackage[margin=2cm]{geometry}

\pagestyle{headandfoot}
\firstpageheader{Names}{}{Period:}
\runningheader{Chemistry B}{Reactivity Lab}{2025 January}
\runningheader{}{}{}
\firstpagefooter{}{}{}
\runningfooter{Chemistry}{First Exam}{Page \thepage\ of \numpages}
\runningfootrule


\begin{document}

\section*{LABORATORIO PRÁCTICO}

\subsection*{Explorando la reactividad}

La reactividad es una medida de la facilidad con la que un elemento transfiere electrones a otro elemento durante una reacción química. La reacción entre el sodio y el hidrógeno es bastante rápida y se caracteriza por un burbujeo vigoroso de la
solución. Otros elementos también reaccionan con el hidrógeno, pero con mayor vigor que el sodio. Otros reaccionan con menor vigor que el sodio. De hecho, puedes clasificar estos elementos en una serie según su reactividad relativa con el hidrógeno.

Al observar el comportamiento de los elementos cuando reaccionan con otras
sustancias, los químicos pueden aprender sobre las propiedades de los elementos y
los átomos que los componen. En este laboratorio, realizarás una investigación
para comparar la reactividad relativa de tres elementos. A partir de los resultados, podrás
sacar conclusiones sobre las propiedades de estos elementos y
cómo se comparan sus estructuras atómicas entre sí.

\paragraph{PREGUNTA DE INVESTIGACIÓN} ¿Qué otros tipos de pruebas utilizan los químicos para
aprender sobre las propiedades de los átomos?

\subsection*{HAGA UNA AFIRMACIÓN}

¿Qué metal cree que reaccionará con mayor fuerza cuando se lo coloca en
ácido clorhídrico? ¿Qué información podría reunir para ayudarlo a hacer
su predicción?

\subsection*{INFORMACIÓN DE SEGURIDAD}

\begin{itemize}
 


    \item Use gafas protectoras contra salpicaduras químicas con ventilación indirecta, un delantal que no sea de látex y guantes de nitrilo durante
las fases de preparación, práctica y desmontaje de la actividad.

\item Use ropa suelta, zapatos cerrados y ate el cabello largo.

\item La reacción entre un metal y el ácido clorhídrico desprende gas hidrógeno. El gas hidrógeno
y los vapores del ácido clorhídrico no deben inhalarse, por lo que estas reacciones deben realizarse dentro de una campana extractora o en una habitación bien ventilada.

\item El ácido clorhídrico (HCl) es un ácido fuerte que es altamente corrosivo para la piel y otros tejidos.
\item Durante una reacción con metales, la inhalación de los vapores puede causar irritación del tracto respiratorio y dificultad para respirar, y el hidrógeno liberado debe mantenerse alejado de las llamas abiertas.

\item Apunte el tubo de ensayo lejos de las personas siempre que contenga reactivos.

\item Tenga cuidado al trabajar con material de vidrio, ya que puede romperse si se cae y cortar la piel.

\item Limpie inmediatamente cualquier líquido que se derrame en el piso para que no se convierta en un peligro de resbalón o caída.

\item Informe a su maestro de inmediato si derrama productos químicos sobre usted, la mesa o el piso.

\item Siga las instrucciones de su maestro para desechar todos los materiales de desecho.

\item Lávese las manos con agua y jabón inmediatamente después de completar esta actividad.


\end{itemize}


\subsection*{MATERIALES}


\begin{itemize}

    \item gafas protectoras contra salpicaduras de productos químicos con ventilación indirecta, delantal sin látex, guantes de nitrilo

    \item aluminio, muestra pequeña 

    \item  ácido clorhídrico, 1 M
\item  magnesio, muestra pequeña
\item  tubos de ensayo (3)
\item  gradilla para tubos de ensayo
\item  zinc, muestra pequeña
 
\end{itemize}


\subsection*{LLEVAR A CABO LA INVESTIGACIÓN}

\begin{enumerate}
    


    \item Coloque un trozo de aluminio en un tubo de ensayo, zinc en un segundo tubo de ensayo y magnesio en un
tercer tubo de ensayo.
\item Coloque aproximadamente de 5 a 10 gotas de ácido clorhídrico 1 M en cada tubo de ensayo.
\item Registre la reactividad relativa (alta, media, baja) de cada elemento en la tabla de datos.
\item Vierta el ácido clorhídrico en un recipiente de desechos designado y enjuague los tubos de ensayo.

\end{enumerate}

\subsection*{RECOPILE DATOS}

Registre la reactividad de cada elemento según sus observaciones.
Elemento Magnesio Aluminio Zinc
Reactividad relativa

\subsection*{ANALIZAR}

\begin{questions}



\question ¿Cómo determinó si una reacción fue vigorosa o no?

\fillwithlines{1in}


\question ¿Qué metal reaccionó con mayor fuerza al colocarlo en el ácido? Proporcione evidencia para respaldar su respuesta.

\fillwithlines{1in}

\end{questions}

\end{document}