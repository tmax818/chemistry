\documentclass[a4paper,12pt]{exam}

\usepackage[margin=2cm]{geometry}

\pagestyle{headandfoot}
\firstpageheader{Names:}{}{Period: }
\runningheader{Chemistry B}{Reactivity Lab}{2025 January}
\runningheader{}{}{}
\firstpagefooter{}{}{}
\runningfooter{Chemistry B}{Reactivity Lab}{Page \thepage\ of \numpages}
\runningfootrule


\begin{document}


\section*{HANDS-ON LAB}

\subsection*{Exploring Reactivity}

Reactivity is a measure of how readily an element transfers electrons to
another element during a chemical reaction. The reaction between sodium
and hydrogen is quite rapid and is marked by a vigorous bubbling of the
solution. Other elements also react with hydrogen but more vigorously than
sodium. Still others react less vigorously than sodium. In fact, you can rank
these elements in a series according to their relative reactivities with
hydrogen.\\

By observing the behavior of elements when they react with other
substances, chemists can learn about the properties of the elements and
the atoms that make them up. In this lab, you will carry out an investigation
to compare the relative reactivities of three elements. From the results, you
will be able to draw conclusions about the properties of these elements and
how their atomic structures compare to one another.\\

\paragraph{RESEARCH QUESTION} What other types of tests do chemists use to
learn about the properties of atoms?

\subsection*{MAKE A CLAIM}

Which metal do you think will react most vigorously when placed in
hydrochloric acid? What information could you gather to help you make
your prediction?

\subsection*{SAFETY INFORMATION}

\begin{itemize}
\item Wear indirectly vented chemical splash goggles, a nonlatex apron, and nitrile gloves during the setup, hands-on, and takedown segments of the activity.

\item Secure loose clothing, wear closed-toe shoes, and tie back long hair.

\item The reaction between a metal and hydrochloric acid gives off hydrogen gas. Hydrogen gas and fumes from hydrochloric acid should not be inhaled, so these reactions should be completed inside a fume hood or in a well-ventilated room.

\item Hydrochloric acid (HCl) is a strong acid that is highly corrosive to skin and other tissues. During a reaction with metals, inhalation of the fumes can cause irritation of the respiratory tract and shortness of breath, and the hydrogen released should be kept away from open flames.

\item  Point the test tube away from people whenever it contains reactants.

\item Use caution when working with glassware, which can shatter if dropped and cut skin.

\item Immediately clean up any liquid spilled on the floor so it does not become a slip/fall hazard.

\item Tell your teacher immediately if you spill chemicals on yourself, the table, or floor.

\item Follow your teacher's instructions for disposing of all waste materials.

\item Wash your hands with soap and water immediately after completing this activity.

\end{itemize}

 \subsection*{MATERIALS}

 \begin{itemize}
    \item  indirectly vented chemical
    \item splash goggles, nonlatex apron,
    \item nitrile gloves
    \item copper, small sample
    \item hydrochloric acid, 1 M
    \item magnesium, small sample
    \item test tubes (3)
    \item test tube rack
    \item zinc, small sample
\end{itemize}

 \subsection*{CARRY OUT THE INVESTIGATION}

 \begin{enumerate}

    \item Place a piece of copper in one test tube, zinc in a second test tube, and magnesium in a third test tube.
    \item Place about 5–10 drops of 1 M hydrochloric acid in each test tube.
    \item Record the relative reactivity (high, medium, low) of each element in the data table.
    \item Pour the hydrochloric acid into a designated waste container, and rinse the test tubes.

\end{enumerate}

\subsection*{COLLECT DATA}

Record the reactivity of each element based on your observations.
\begin{center}
    \begin{tabular}{|p {3.5cm}|p {3.5cm}|p {3.5cm}|p {3.5cm}|}
        \hline
        Element & Magnesium & copper & Zinc \\ 
     \hline
     Relative reactivity  & \; & \; & \;        \\
     \hline    
    \end{tabular}
    \end{center}



\subsection*{ANALYZE}

\begin{questions}
    
\question How did you determine whether a reaction was vigorous or not?

\fillwithlines{1in}

\question Which metal reacted most vigorously when placed in the acid? Give evidence to support your
answer.


\fillwithlines{1in}

\end{questions}
\end{document}