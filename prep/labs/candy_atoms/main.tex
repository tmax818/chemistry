\documentclass[a4paper,12pt]{exam}

\usepackage[margin=2cm]{geometry}
\usepackage[version=4]{mhchem}

\pagestyle{headandfoot}
\firstpageheader{Names:}{}{Period: }
\runningheader{Chemistry B}{Sweet Atoms}{2025 January}
\runningheader{}{}{}
\firstpagefooter{}{}{}
\runningfooter{Chemistry B}{Sweet Atoms}{Page \thepage\ of \numpages}
\runningfootrule


\begin{document}


\section*{Sweet Model of the Atom}

\subsection*{Purpose}
The purpose of this lab is to visually see how atoms, isotopes and ions are put together.


 \subsection*{MATERIALS}

 \begin{itemize}
    \item Skittles
    \item Nerds
    \item Paper
    \item Glue(optional)
\end{itemize}

 \subsection*{Procedure}

 \begin{enumerate}

    \item Establish which subatomic particle each color of candy represents.\\
        Electrons:\\
        Protons:\\
        Neutrons:
    \item For the following examples, be sure to make them using your candies. When you have made the model with your candy, draw the atom on your paper.
    Be sure to answer the questions as you complete the activity.

\end{enumerate}




\subsection*{Atoms}

\begin{questions}
    
\question \ce{^{1}_{}H}
\vspace{8cm}

\pagebreak
\question \ce{^{4}_{2}He}
\vspace{8cm}


\question \ce{^{7}_{3}Li}
\vspace{8cm}

\question \ce{^{7}_{3}Li+}
\vspace{8cm}


\end{questions}
\end{document}